\documentclass[oneside]{article}

% Copyright (c) 2018 Michael Heilmann. All rights reserved.

\makeatletter

% "(Get|Set)Author".
% The name of the author.
\def\SetAuthor#1{\gdef\@author{#1}}
\def\@author{\@latex@warning@no@line{No author given}}
\def\GetAuthor{\@author}

% "(Get|Set)Email".
% The email (address) of the author.
\def\SetEmail#1{\gdef\@email{#1}}
\def\@email{\@latex@warning@no@line{No email given}}
\def\GetEmail{\@email}

% "(Get|Set)Organization".
% The organization the library is published by.
\def\SetOrganization#1{\gdef\@organization{#1}}
\def\@organization{\@latex@warning@no@line{No organization given}}
\def\GetOrganization{\@organization}

% "(Get|Set)LibraryName".
% The name of the library.
\def\SetLibraryName#1{\gdef\@libraryName{#1}}
\def\@libraryName{\@latex@warning@no@line{No library name given}}
\def\GetLibraryName{\@libraryName}

% "(Get|Set)LibraryVersion".
% The version of the library.
\def\SetLibraryVersion#1{\gdef\@libraryVersion{#1}}
\def\@libraryVersion{\@latex@warning@no@line{No library version given}}
\def\GetLibraryVersion{\@libraryVersion}

% "(Get|Set)LibraryRepository".
% The repository (url) of the library.
\def\SetLibraryRepository#1{\gdef\@libraryRepository{#1}}
\def\@libraryrepository{\@latex@warning@no@line{No library repository given}}
\def\GetLibraryRepository{\@libraryRepository}

% Define "maketitle".
\def\maketitle{%
  \noindent\makebox[\textwidth]{%
	\uppercase{{\GetOrganization} {\GetLibraryName} {\GetLibraryVersion}}%
	\hfill
	\uppercase{{\GetAuthor} {(\href{mailto:\GetEmail}{\GetEmail})}}%
  }%
}

\makeatother


\SetOrganization{Primordial Machine's}
\SetLibraryName{Arithmetic Functors Library}
\SetLibraryVersion{v1.3}
\SetLibraryRepository{https://github.com/primordialmachine/arithmetic-functors}
\SetAuthor{Michael Heilmann}
\SetEmail{michaelheilmann@primordialmachine.com}

\begin{document}
\maketitle
\tableofcontents
\section{Synopsis}
C++ 17 library providing arithmetic functors.
The library is made available publicly on
\href{\GetLibraryRepository}{Github}
under the
\href{\GetLibraryRepository/blob/master/LICENSE}{MIT License}.

\section{Limitations and Restrictions}
The library officially only supports Visual Studio 2017 and Windows 10.

\section{Introductory example}
\textit{\color{orange}This library does not provide any examples yet.}
%Examples are located in the \href{\GetLibraryRepository/blob/master/examples}{examples} directory.

\section{Building under Visual Studio 2017}
\begin{enumerate}
\item Open the solution \texttt{solution.sln} in Microsoft Visual Studio 2017.
\item Batch build everything.
\item The folder \texttt{packages} contains the distribution of the library i.e. include files and the
      static libraries for
  \begin{enumerate}
    \item the platforms \texttt{Win32} and \texttt{x64} and
    \item configurations \texttt{Release} and \texttt{Debug}.
  \end{enumerate}
\item Copy the contents of the \verb+packages+ folder into a directory. Let
      \verb+[library home]+ be a placeholder denoting the path by which that folder
      can be referenced from your project.
\item Add
  \begin{enumerate}
    \item the include path
\begin{verbatim}
[library home]/primordialmachine/arithmetic-functors/
$(PlatformTarget.toLower())/$(Configuration.toLower())/includes
\end{verbatim}
	and
    \item the library path
\begin{verbatim}
[library home]/primordialmachine/arithmetic-functors/
$(PlatformTarget.toLower())/$(Configuration.toLower())/libraries
\end{verbatim}
    to your project.
\end{enumerate}
\item Link your project with the library \verb+arithmetic-functors.lib+.
\item Add the include directive \verb+#include "primordialmachine/arithmetic_functors/include.hpp"+ where appropriate.
\item You can now use the functionality provided by the library.
\end{enumerate}

\section{Library Interface Documentation}

\subsection{\texttt{namespace primordialmachine}}
The namespace this library is adding its declarations/definitions to.
The added namespace elements are documented below.

\subsection{\textit{concept BinaryFunctor}}
A \textit{BinaryFunctor} is a function object with three template arguments \verb+LEFT_OPERAND+,
\verb+RIGHT_OPERAND+, and \verb+ENABLED+. Specializations of this type must provide a
constant \verb+operator()+ with two parameters \verb+left_operand+ of type \verb+LEFT_OPERAND+
(or a cv-qualified variant of that) and \verb+right_operand+ of type \verb+RIGHT_OPERAND+
(or a cv-qualified variant of that). The return value of that operand shall represent the
result of the operation \verb+left_operand OP right_operand+ where OP is the binary operation
the \textit{BinaryFunctor} represents.\\

\noindent{}The default type of \verb+ENABLED+ is \verb+void+. Specializations of this type may use
the \verb+ENABLED+ argument to perform \textit{SFINAE}.

\subsubsection{\texttt{struct binary\_plus\_functor}}
A \textit{BinaryFunctor} representing the binary operation \verb|left_operand + right_operand|.

\subsubsection{\texttt{struct binary\_minus\_functor}}
A \textit{BinaryFunctor} representing the binary expression \verb|left_operand - right_operand|.

\subsubsection{\texttt{struct binary\_star\_functor}}
A \textit{BinaryFunctor} representing the binary expression \verb|left_operand * right_operand|.

\subsubsection{\texttt{struct binary\_slash\_functor}}
A \textit{BinaryFunctor} representing the binary expression \verb|left_operand / right_operand|.

\subsection{\textit{concept UnaryFunctor}}
A \textit{UnaryFunctor} is a function object with two template arguments \verb+LEFT_OPERAND+,
\verb+RIGHT_OPERAND+, and \verb+ENABLED+. Specializations of this type must provide a
constant \verb+operator()+ with one parameter \verb+operand+ of type \verb+OPERAND+
(or a cv-qualified variant of that). The return value of that operand shall represent the
result of the operation \verb+OP operand+ where OP is the unary operation
the \textit{UnaryFunctor} represents.\\

\noindent{}The default type of \verb+ENABLED+ is \verb+void+. Specializations of this type may use
the \verb+ENABLED+ argument to perform \textit{SFINAE}.

\subsubsection{\texttt{struct unary\_plus\_functor}}
A \textit{UnaryFunctor} representing the unary expression \verb|+operand|.

\subsubsection{\texttt{struct unary\_minus\_functor}}
A \textit{UnaryFunctor} representing the unary expression \verb|-operand|.

\end{document}
