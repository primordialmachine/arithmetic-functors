\documentclass[oneside]{book}

% Copyright (c) 2018 Michael Heilmann. All rights reserved.

\makeatletter

% "(Get|Set)Author".
% The name of the author.
\def\SetAuthor#1{\gdef\@author{#1}}
\def\@author{\@latex@warning@no@line{No author given}}
\def\GetAuthor{\@author}

% "(Get|Set)Email".
% The email (address) of the author.
\def\SetEmail#1{\gdef\@email{#1}}
\def\@email{\@latex@warning@no@line{No email given}}
\def\GetEmail{\@email}

% "(Get|Set)Organization".
% The organization the library is published by.
\def\SetOrganization#1{\gdef\@organization{#1}}
\def\@organization{\@latex@warning@no@line{No organization given}}
\def\GetOrganization{\@organization}

% "(Get|Set)LibraryName".
% The name of the library.
\def\SetLibraryName#1{\gdef\@libraryName{#1}}
\def\@libraryName{\@latex@warning@no@line{No library name given}}
\def\GetLibraryName{\@libraryName}

% "(Get|Set)LibraryVersion".
% The version of the library.
\def\SetLibraryVersion#1{\gdef\@libraryVersion{#1}}
\def\@libraryVersion{\@latex@warning@no@line{No library version given}}
\def\GetLibraryVersion{\@libraryVersion}

% "(Get|Set)LibraryRepository".
% The repository (url) of the library.
\def\SetLibraryRepository#1{\gdef\@libraryRepository{#1}}
\def\@libraryrepository{\@latex@warning@no@line{No library repository given}}
\def\GetLibraryRepository{\@libraryRepository}

% Define "maketitle".
\def\maketitle{%
  \noindent\makebox[\textwidth]{%
	\uppercase{{\GetOrganization} {\GetLibraryName} {\GetLibraryVersion}}%
	\hfill
	\uppercase{{\GetAuthor} {(\href{mailto:\GetEmail}{\GetEmail})}}%
  }%
}

\makeatother



\SetOrganization{Primordial Machine's}
\SetLibraryName{Arithmetic Functors Library}

\SetLibraryIncludeFileName{include.hpp}
\SetLibraryIncludesDirectoryPath{primordialmachine/arithmetic-functors/\newline\$(PlatformTarget.toLower())/\$(Configuration.toLower())/includes}

\SetLibraryIncludeDirectiveFilePath{primordialmachine/arithmetic\_functors/include.hpp}

\SetLibraryStaticLibrariesDirectoryPath{primordialmachine/arithmetic-functors/\newline\$(PlatformTarget.toLower())/\$(Configuration.toLower())/libraries}
\SetLibraryStaticLibraryFileName{arithmetic-functors.lib}

\SetLibraryVersion{v1.6}
\SetLibraryRepository{https://github.com/primordialmachine/arithmetic-functors}
\SetAuthor{Michael Heilmann}
\SetEmail{michaelheilmann@primordialmachine.com}

\SetDocumentType{User Manual}

\begin{document}

\frontmatter

\begin{titlepage}
\maketitle
\end{titlepage}

\tableofcontents
\addtocontents{toc}{\protect\thispagestyle{empty}}
\pagenumbering{gobble}

\mainmatter

\chapter{Synopsis}
C++ 17 library providing arithmetic functors and arithmetic errors and exceptions.
The library is made available publicly on
\href{\GetLibraryRepository}{Github}
under the
\href{\GetLibraryRepository/blob/master/LICENSE}{MIT License}.

\chapter{Limitations and Restrictions}
The library officially only supports Visual Studio 2017 and Windows 10.

\chapter{Requirements}
This library depends on the \href{https://github.com/primordialmachine/errors}{Primordial Machine's Errors Library}.

\chapter{Introductory example}
\textit{\color{orange}This library does not provide any examples yet.}
%Examples are located in the \href{\GetLibraryRepository/blob/master/examples}{examples} directory.

% Copyright (c) 2018 Michael Heilmann. All rights reserved.
\chapter{Building under Visual Studio 2017}
\begin{enumerate}
\item Open the solution \texttt{solution.sln} in Microsoft Visual Studio 2017.
\item Batch build everything.
\item The folder \texttt{packages} contains the distribution of the library i.e. include files and the
      static libraries for
  \begin{enumerate}
    \item the platform targets \texttt{x86} and \texttt{x64} and
    \item configurations \texttt{Release} and \texttt{Debug}.
  \end{enumerate}
\item Copy the contents of the \verb+packages+ folder into a directory. Let
      \verb+[library home]+ be a placeholder denoting the path by which that folder
      can be referenced from your project.
\item Add
  \begin{enumerate}
    \item the include path
\texttt{[library home]/\GetLibraryIncludesDirectoryPath}
	and
    \item the library path
\texttt{[library home]/\GetLibraryStaticLibrariesDirectoryPath}
    to your project.
\end{enumerate}
\item Link your project with the library \texttt{\GetLibraryStaticLibraryFileName}.
\item Add the include directive \texttt{\#include "{}\GetLibraryIncludeDirectiveFilePath"{}} where appropriate.
\item You can now use the functionality provided by the library.
\end{enumerate}


\chapter{Library Interface Documentation}

\section{\texttt{namespace primordialmachine}}
The namespace this library is adding its declarations/definitions to.
The added namespace elements are documented below.

%%%%%%%%%%%%%%%%%%%%%%%%%%%%%%%%%%%%%%%%%%%%%%%%%%%%%%%%%%%%%%%%%%%%%%%%%%%%%%%%%%%%%%%%%%%%%%%%%%%
\section{\textit{concept Functor}}
A functor is a struct type providing a constant  \texttt{operator()}.
That operator shall be qualified as \texttt{noexcept}     if possible
and shall be qualified as \texttt{constexpr} if possible.  The return
type of \texttt{operator()} shall be of type \texttt{result}    (or a
cv-qualified variant of that).\newline\noindent{}The functor    shall
provide the member type definition \texttt{result\_type} denoting the
type \texttt{result}.

%%%%%%%%%%%%%%%%%%%%%%%%%%%%%%%%%%%%%%%%%%%%%%%%%%%%%%%%%%%%%%%%%%%%%%%%%%%%%%%%%%%%%%%%%%%%%%%%%%%
\section{\textit{concept BinaryFunctor}}
A \textit{BinaryFunctor} is a \textit{Functor}.
Its \texttt{operator()} has two parameters \texttt{left\_operand} of type
\texttt{left\_operand\_type} (or a cv-qualified variant of that)      and 
\texttt{right\_operand} of type \texttt{right\_operand\_type} (or       a
cv-qualified variant of that).\newline

\noindent{}The functor shall provide the member type              definitions
\texttt{left\_operand\_type}  denoting the type \texttt{left\_operand}    and
\texttt{right\_operand\_type} denoting the type      \texttt{right\_operand}.
\newline

\section{\textit{concept BinaryFunctorBase}}
A  \textit{BinaryFunctorBase} is a template struct type of   name \textit{name}
with three template parameters \texttt{LEFT\_OPERAND}, \texttt{RIGHT\_OPERAND},
and \texttt{ENABLED}. The default value of \texttt{ENALBED} is   \texttt{void}.
Specializations of this template may use \texttt{ENABLED} to    perform SFINAE.
A possible implementation is
\texttt{\newline
\noindent{}template\textlangle typename LEFT\_OPERAND, typename RIGHT\_OPERAND,
typename ENABLED = void\textrangle\newline\noindent{}struct \textit{name};    }
\newline

\noindent{}Its partial specializations are \textit{BinaryFunctors}.
%%%%%%%%%%%%%%%%%%%%%%%%%%%%%%%%%%%%%%%%%%%%%%%%%%%%%%%%%%%%%%%%%%%%%%%%%%%%%%%%%%%%%%%%%%%%%%%%%%%
\section{\textit{concept UnaryFunctor}}
An \textit{UnaryFunctor} is \textit{Functor}.
Its \texttt{operator()} has one parameter \texttt{operand}  of type
\texttt{operand\_type} (or a cv-qualified variant of that).\newline

\noindent{}The functor shall provide the member type definition
\texttt{operand\_type} denoting the type      \texttt{operand}.

\section{\textit{concept UnaryFunctorBase}}
An \textit{UnaryFunctorBase} is a template struct type of     name \textit{name}
with two  template parameters \texttt{OPERAND} and \texttt{ENABLED}. The default
value of \texttt{ENALBED} is \texttt{void}. Specializations of this template may
use \texttt{ENABLED} to perform SFINAE.
A possible implementation is
\texttt{\newline
\noindent{}template\textlangle typename OPERAND,
typename ENABLED = void\textrangle\newline\noindent{}struct \textit{name};    }
\newline

\noindent{}Its partial specializations are \textit{UnaryFunctors}.


%%%%%%%%%%%%%%%%%%%%%%%%%%%%%%%%%%%%%%%%%%%%%%%%%%%%%%%%%%%%%%%%%%%%%%%%%%%%%%%%%%%%%%%%%%%%%%%%%%%
\section{\texttt{struct binary\_plus\_functor}}
A \textit{BinaryFunctorBase} representing the binary operation $\textit{left\_operand} + \textit{right\_operand}$.\newline

\noindent{}This library provides \textit{BinaryFunctor}       specializations for
\texttt{LEFT\_OPERAND} and \texttt{RIGHT\_OPERAND} being     floating point types
which compute  the sum    of  \texttt{left\_operand} and \texttt{right\_operand}.
\texttt{result\_type} is \texttt{std::common\_type\_t  \textlangle LEFT\_OPERAND,
RIGHT\_OPERAND\textrangle}.

%%%%%%%%%%%%%%%%%%%%%%%%%%%%%%%%%%%%%%%%%%%%%%%%%%%%%%%%%%%%%%%%%%%%%%%%%%%%%%%%%%%%%%%%%%%%%%%%%%%
\section{\texttt{struct binary\_minus\_functor}}
A \textit{BinaryFunctorBase} representing the binary expression $\textit{left\_operand} - \textit{right\_operand}$.\newline

\noindent{}This library provides \textit{BinaryFunctor}         specializations for
\texttt{LEFT\_OPERAND} and   \texttt{RIGHT\_OPERAND} being     floating point types
which compute the difference of \texttt{left\_operand} and \texttt{right\_operand}.
\texttt{result\_type} is \texttt{std::common\_type\_t  \textlangle   LEFT\_OPERAND,
RIGHT\_OPERAND\textrangle}.

%%%%%%%%%%%%%%%%%%%%%%%%%%%%%%%%%%%%%%%%%%%%%%%%%%%%%%%%%%%%%%%%%%%%%%%%%%%%%%%%%%%%%%%%%%%%%%%%%%%
\section{\texttt{struct binary\_star\_functor}}
A \textit{BinaryFunctorBase} representing the binary expression $\textit{left\_operand} \cdot \textit{right\_operand}$.\newline

\noindent{}This library provides \textit{BinaryFunctor}         specializations for
\texttt{LEFT\_OPERAND} and   \texttt{RIGHT\_OPERAND} being     floating point types
which compute the difference of \texttt{left\_operand} and \texttt{right\_operand}.
\texttt{result\_type} is \texttt{std::common\_type\_t  \textlangle   LEFT\_OPERAND,
RIGHT\_OPERAND\textrangle}.

%%%%%%%%%%%%%%%%%%%%%%%%%%%%%%%%%%%%%%%%%%%%%%%%%%%%%%%%%%%%%%%%%%%%%%%%%%%%%%%%%%%%%%%%%%%%%%%%%%%
\section{\texttt{struct binary\_slash\_functor}}
A \textit{BinaryFunctorBase} representing the binary expression $\textit{left\_operand} / \textit{right\_operand}$.\newline

\noindent{}This library provides \textit{BinaryFunctor}         specializations for
\texttt{LEFT\_OPERAND} and   \texttt{RIGHT\_OPERAND} being     floating point types
which compute the difference of \texttt{left\_operand} and \texttt{right\_operand}.
\texttt{result\_type} is \texttt{std::common\_type\_t  \textlangle   LEFT\_OPERAND,
RIGHT\_OPERAND\textrangle}.

%%%%%%%%%%%%%%%%%%%%%%%%%%%%%%%%%%%%%%%%%%%%%%%%%%%%%%%%%%%%%%%%%%%%%%%%%%%%%%%%%%%%%%%%%%%%%%%%%%%
\section{\texttt{struct unary\_plus\_functor}}
A \textit{UnaryFunctorBase} representing the unary expression $+\textit{operand}$.\newline

\noindent{}This library provides \textit{UnaryFunctor}            specializations for
\texttt{OPERAND} being a floating point type which compute the arithmetic affirmation
of \texttt{operand}. \texttt{result\_type} is \texttt{std::common\_type\_t\textlangle
OPERAND\textrangle}.

%%%%%%%%%%%%%%%%%%%%%%%%%%%%%%%%%%%%%%%%%%%%%%%%%%%%%%%%%%%%%%%%%%%%%%%%%%%%%%%%%%%%%%%%%%%%%%%%%%%
\section{\texttt{struct unary\_minus\_functor}}
A \textit{UnaryFunctorBase} representing the unary expression $-\textit{operand}$.\newline

\noindent{}This library provides \textit{UnaryFunctor}            specializations for
\texttt{OPERAND} being a floating point type which compute the arithmetic    negation
of \texttt{operand}. \texttt{result\_type} is \texttt{std::common\_type\_t\textlangle
OPERAND\textrangle}.

%%%%%%%%%%%%%%%%%%%%%%%%%%%%%%%%%%%%%%%%%%%%%%%%%%%%%%%%%%%%%%%%%%%%%%%%%%%%%%%%%%%%%%%%%%%%%%%%%%%
\section{\texttt{division\_by\_zero\_error}}
An error (as specified in Primordial Machine's Errors library) indicating a division by zero.

\section{\textit{concept Error}}
\textit{\color{orange}This section needs to be moved in Primordial Machine's Error library.}
\noindent\textit{\color{orange}[[}\\
An \textit{Error} type is a \texttt{error} or derived type.\\
\noindent\textit{\color{orange}]]}

\section{\textit{concept Exception}}
\textit{\color{orange}This section needs to be moved in Primordial Machine's Error library.}
\noindent\textit{\color{orange}[[}\\
An \textit{Exception} type is a \texttt{exception} or derived type.
Such an object must retain the properties of being
\begin{enumerate}
  \item noexcept copy/move constructible and
  \item noexcept copy/move assignable.
\end{enumerate}
\noindent\textit{\color{orange}]]}

\section{\texttt{negative\_overflow\_error}}
An \textit{Error} type (as specified in Primordial Machine's Errors library) indicating a negative overflow.
It raises its corresponding \texttt{negative\_overflow\_exception}.
The term negative overflow refers to the situation in which an arithmetic value is non-positive and
too big (in terms of its magnitude) to be represented. Its constructor takes a single argument which
is a \texttt{error\_position} value.

\section{\texttt{positive\_overflow\_error}}
An \textit{Error} type (as specified in Primordial Machine's Errors library) indicating a positive overflow.
It raises its corresponding \texttt{negative\_overflow\_exception}.
The term positive overflow refers to the situation in which an arithmetic value is non-negative and
too big (in terms of its magnitude) to be represented. Its constructor takes a single argument which
is a \texttt{error\_position} value.

\section{\texttt{underflow\_error}}
An \textit{Error} type (as specified in Primordial Machine's Errors library) indicating an underflow.
It raises its corresponding \texttt{underflow\_exception}.
The term underflow refers to the situation in which an arithmetic value is too small (in terms of
its magnitude) to be represented. Its constructor takes a single argument which
is a \texttt{error\_position} value.

\end{document}
