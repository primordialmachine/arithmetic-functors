%%%%%%%%%%%%%%%%%%%%%%%%%%%%%%%%%%%%%%%%%%%%%%%%%%%%%%%%%%%%%%%%%%%%%%%%%%%%%%%%%%%%%%%%%%%%%%%%%%%
%
% Primordial Machine's Arithmetic Functors Library
% Copyright (C) 2017-2019 Michael Heilmann
%
% This software is provided 'as-is', without any express or implied warranty.
% In no event will the authors be held liable for any damages arising from the
% use of this software.
%
% Permission is granted to anyone to use this software for any purpose,
% including commercial applications, and to alter it and redistribute it
% freely, subject to the following restrictions:
%
% 1. The origin of this software must not be misrepresented;
%    you must not claim that you wrote the original software.
%    If you use this software in a product, an acknowledgment
%    in the product documentation would be appreciated but is not required.
%
% 2. Altered source versions must be plainly marked as such,
%    and must not be misrepresented as being the original software.
%
% 3. This notice may not be removed or altered from any source distribution.
%
%%%%%%%%%%%%%%%%%%%%%%%%%%%%%%%%%%%%%%%%%%%%%%%%%%%%%%%%%%%%%%%%%%%%%%%%%%%%%%%%%%%%%%%%%%%%%%%%%%%

\documentclass[oneside]{book}

\input{common}
\setlength{\parindent}{0mm}

\SetOrganization{Primordial Machine's}
\SetLibraryName{Arithmetic Functors Library}

\SetLibraryIncludeFileName{include.hpp}
\SetLibraryIncludesDirectoryPath{primordialmachine/arithmetic-functors/\newline\$(PlatformTarget.toLower())/\$(Configuration.toLower())/includes}

\SetLibraryIncludeDirectiveFilePath{primordialmachine/arithmetic\_functors/include.hpp}

\SetLibraryStaticLibrariesDirectoryPath{primordialmachine/arithmetic-functors/\newline\$(PlatformTarget.toLower())/\$(Configuration.toLower())/libraries}
\SetLibraryStaticLibraryFileName{arithmetic-functors.lib}

\SetLibraryVersion{v2.0}
\SetLibraryRepository{https://github.com/primordialmachine/math-arithmetic-functors}
\SetAuthor{Michael Heilmann}
\SetEmail{michaelheilmann@primordialmachine.com}

\SetDocumentType{User Manual}

\begin{document}

\frontmatter

\begin{titlepage}
\maketitle
\end{titlepage}

\tableofcontents
\addtocontents{toc}{\protect\thispagestyle{empty}}
\pagenumbering{gobble}

\mainmatter

\chapter{Synopsis}
C++ 17 library providing arithmetic functors and arithmetic errors and exceptions.
The library is made available publicly on
\href{\GetLibraryRepository}{Github}
under the
\href{\GetLibraryRepository/blob/master/LICENSE}{MIT License}.

\chapter{Limitations and Restrictions}
The library officially only supports Visual Studio 2017 and Windows 10.

\chapter{Requirements}
This library depends on the \href{https://github.com/primordialmachine/errors}{Primordial Machine's Errors Library}.

\chapter{Introductory example}
\textit{\color{orange}This library does not provide any examples yet.}
%Examples are located in the \href{\GetLibraryRepository/blob/master/examples}{examples} directory.

\input{building_visual_studio_2017}

\chapter{Library Interface Documentation}

\section{\texttt{namespace primordialmachine}}
The namespace this library is adding its declarations/definitions to.
The added namespace elements are documented below.

\input{functors}

%%%%%%%%%%%%%%%%%%%%%%%%%%%%%%%%%%%%%%%%%%%%%%%%%%%%%%%%%%%%%%%%%%%%%%%%%%%%%%%%%%%%%%%%%%%%%%%%%%%
\section{\texttt{binary\_plus\_functor} (struct)}
A \textit{BinaryFunctorBase} representing the binary operation $\textit{left\_operand} + \textit{right\_operand}$.\newline

\subsection{\texttt{binary\_plus} (function)}
This library provides a function to invoke the \texttt{binary\_plus\_functor}.
A possible implementation is given by\newline
\texttt{
template\textlangle typename A, typename B\textrangle\newline
auto binary\_plus(const A\& a, const B\& b) -> decltype(binary\_plus\_functor\textlangle A, B\textrangle()(a, b))\newline
\{ return binary\_plus\_functor\textlangle A, B\textrangle()(a, b); \}
}

\subsection{\texttt{operator+} (binary operator)}
This library provides a binary plus operator overload to invoke the \texttt{binary\_plus\_functor}.
A possible implementation is given by\newline
\texttt{
template\textlangle typename A, typename B\textrangle\newline
auto operator+(const A\& a, const B\& b) -> decltype(binary\_plus(a, b))\newline
\{ return binary\_plus(a, b); \}
}

%%%%%%%%%%%%%%%%%%%%%%%%%%%%%%%%%%%%%%%%%%%%%%%%%%%%%%%%%%%%%%%%%%%%%%%%%%%%%%%%%%%%%%%%%%%%%%%%%%%
\section{\texttt{binary\_minus\_functor} (struct)}
A \textit{BinaryFunctorBase} representing the binary expression $\textit{left\_operand} - \textit{right\_operand}$.\newline

\subsection{\texttt{binary\_minus} (function)}
This library provides a function to invoke the \texttt{binary\_minus\_functor}.
A possible implementation is given by\newline
\texttt{
template\textlangle typename A, typename B\textrangle\newline
auto binary\_minus(const A\& a, const B\& b) -> decltype(binary\_minus\_functor\textlangle A, B\textrangle()(a, b))\newline
\{ return binary\_minus\_functor\textlangle A, B\textrangle()(a, b); \}
}

\subsection{\texttt{operator-} (binary operator)}
This library provides a binary minus operator overload to invoke the \texttt{binary\_minus\_functor}.
A possible implementation is given by\newline
\texttt{
template\textlangle typename A, typename B\textrangle\newline
auto operator-(const A\& a, const B\& b) -> decltype(binary\_minus(a, b))\newline
\{ return binary\_minus(a, b); \}
}

%%%%%%%%%%%%%%%%%%%%%%%%%%%%%%%%%%%%%%%%%%%%%%%%%%%%%%%%%%%%%%%%%%%%%%%%%%%%%%%%%%%%%%%%%%%%%%%%%%%
\section{\texttt{binary\_star\_functor} (struct)}
A \textit{BinaryFunctorBase} representing the binary expression $\textit{left\_operand} \cdot \textit{right\_operand}$.\newline

\subsection{\texttt{binary\_star} (function)}
This library provides a function to invoke the \texttt{binary\_star\_functor}.
A possible implementation is given by\newline
\texttt{
template\textlangle typename A, typename B\textrangle\newline
auto binary\_star(const A\& a, const B\& b) -> decltype(binary\_star\_functor\textlangle A, B\textrangle()(a, b))\newline
\{ return binary\_star\_functor\textlangle A, B\textrangle()(a, b); \}
}

\subsection{\texttt{operator*} (binary operator)}
This library provides a binary star operator overload to invoke the \texttt{binary\_star\_functor}.
A possible implementation is given by\newline
\texttt{
template\textlangle typename A, typename B\textrangle\newline
auto operator*(const A\& a, const B\& b) -> decltype(binary\_star(a, b))\newline
\{ return binary\_star(a, b); \}
}

%%%%%%%%%%%%%%%%%%%%%%%%%%%%%%%%%%%%%%%%%%%%%%%%%%%%%%%%%%%%%%%%%%%%%%%%%%%%%%%%%%%%%%%%%%%%%%%%%%%
\section{\texttt{binary\_slash\_functor} (struct)}
A \textit{BinaryFunctorBase} representing the binary expression $\textit{left\_operand} / \textit{right\_operand}$.\newline

\subsection{\texttt{binary\_slash} (function)}
This library provides a function to invoke the \texttt{binary\_slash\_functor}.
A possible implementation is given by\newline
\texttt{
template\textlangle typename A, typename B\textrangle\newline
auto binary\_slash(const A\& a, const B\& b) -> decltype(binary\_slash\_functor\textlangle A, B\textrangle()(a, b))\newline
\{ return binary\_slash\_functor\textlangle A, B\textrangle()(a, b); \}
}

\subsection{\texttt{operator/} (binary operator)}
This library provides a binary star operator overload to invoke the \texttt{binary\_star\_functor}.
A possible implementation is given by\newline
\texttt{
template\textlangle typename A, typename B\textrangle\newline
auto operator/(const A\& a, const B\& b) -> decltype(binary\_slash(a, b))\newline
\{ return binary\_slash(a, b); \}
}

%%%%%%%%%%%%%%%%%%%%%%%%%%%%%%%%%%%%%%%%%%%%%%%%%%%%%%%%%%%%%%%%%%%%%%%%%%%%%%%%%%%%%%%%%%%%%%%%%%%
\section{\texttt{unary\_plus\_functor} (struct)}
A \textit{UnaryFunctorBase} representing the unary expression $+\textit{operand}$.\newline

\subsection{\texttt{unary\_plus} (function)}
This library provides a function to invoke the \texttt{unary\_plus\_functor}.
A possible implementation is given by\newline
\texttt{
template\textlangle typename A\textrangle\newline
auto unary\_plus(const A\& a) -> decltype(unary\_plus\_functor\textlangle A\textrangle()(a))\newline
\{ return unary\_plus\_functor\textlangle A\textrangle()(a); \}
}

\subsection{\texttt{operator+} (unary operator)}
This library provides a unary plus operator overload to invoke the \texttt{unary\_plus\_functor}.
A possible implementation is given by\newline
\texttt{
template\textlangle typename A\textrangle\newline
auto operator+(const A\& a) -> decltype(unary\_plus(a))\newline
\{ return unary\_plus(a); \}
}

%%%%%%%%%%%%%%%%%%%%%%%%%%%%%%%%%%%%%%%%%%%%%%%%%%%%%%%%%%%%%%%%%%%%%%%%%%%%%%%%%%%%%%%%%%%%%%%%%%%
\section{\texttt{unary\_minus\_functor} (struct)}
A \textit{UnaryFunctorBase} representing the unary expression $-\textit{operand}$.\newline

\subsection{\texttt{unary\_minus} (function)}
This library provides a function to invoke the \texttt{unary\_minus\_functor}.
A possible implementation is given by\newline
\texttt{
template\textlangle typename A\textrangle\newline
auto unary\_minus(const A\& a) -> decltype(unary\_minus\_functor\textlangle A\textrangle()(a))\newline
\{ return unary\_minus\_functor\textlangle A\textrangle()(a); \}
}

\subsection{\texttt{operator+} (unary operator)}
This library provides a unary minus operator overload to invoke the \texttt{unary\_minus\_functor}.
A possible implementation is given by\newline
\texttt{
template\textlangle typename A\textrangle\newline
auto operator-(const A\& a) -> decltype(unary\_minus(a))\newline
\{ return unary\_minus(a); \}
}

%%%%%%%%%%%%%%%%%%%%%%%%%%%%%%%%%%%%%%%%%%%%%%%%%%%%%%%%%%%%%%%%%%%%%%%%%%%%%%%%%%%%%%%%%%%%%%%%%%%
%
% Primordial Machine's Arithmetic Functors Library
% Copyright (c) 2017-2019 Michael Heilmann
%
% This software is provided 'as-is', without any express or implied warranty.
% In no event will the authors be held liable for any damages arising from the
% use of this software.
%
% Permission is granted to anyone to use this software for any purpose,
% including commercial applications, and to alter it and redistribute it
% freely, subject to the following restrictions:
%
% 1. The origin of this software must not be misrepresented;
%    you must not claim that you wrote the original software.
%    If you use this software in a product, an acknowledgment
%    in the product documentation would be appreciated but is not required.
%
% 2. Altered source versions must be plainly marked as such,
%    and must not be misrepresented as being the original software.
%
% 3. This notice may not be removed or altered from any source distribution.
%
%%%%%%%%%%%%%%%%%%%%%%%%%%%%%%%%%%%%%%%%%%%%%%%%%%%%%%%%%%%%%%%%%%%%%%%%%%%%%%%%%%%%%%%%%%%%%%%%%%%

\section{\texttt{plus\_assignment\_functor} (struct)}
A \textit{FunctorBase} (see \cite{functors}) with two template parameters \texttt{A} and \texttt{B} followed by the \texttt{ENABLED} template parameter.
Its specializations are \textit{Functor}s representing the operation $\textit{a += b}$.

\subsection{\texttt{plus\_assignment} (function)}
This library provides a function to invoke the \texttt{plus\_assignment\_functor}.
A possible implementation is given by\newline
\texttt{template\textlangle typename A, typename B\textrangle\newline
auto\newline
plus\_assignment(A\& a, const B\& b)\newline
-> decltype(plus\_assignment\_functor\textlangle A, B\textrangle()(a, b))\newline
\{ return plus\_assignment\_functor\textlangle A, B\textrangle()(a, b); \}}

\subsection{\texttt{operator+=} (compound operator)}
This library provides a compound plus operator overload to invoke the \texttt{plus\_assignment\_functor}.
A possible implementation is given by\newline
\texttt{template\textlangle typename A, typename B\textrangle\newline
auto\newline
operator+=(A\& a, const B\& b)\newline
-> decltype(plus\_assignment(a, b))\newline
\{ return plus\_assignment(a, b); \}}

%%%%%%%%%%%%%%%%%%%%%%%%%%%%%%%%%%%%%%%%%%%%%%%%%%%%%%%%%%%%%%%%%%%%%%%%%%%%%%%%%%%%%%%%%%%%%%%%%%%
%
% Primordial Machine's Arithmetic Functors Library
% Copyright (c) 2017-2019 Michael Heilmann
%
% This software is provided 'as-is', without any express or implied warranty.
% In no event will the authors be held liable for any damages arising from the
% use of this software.
%
% Permission is granted to anyone to use this software for any purpose,
% including commercial applications, and to alter it and redistribute it
% freely, subject to the following restrictions:
%
% 1. The origin of this software must not be misrepresented;
%    you must not claim that you wrote the original software.
%    If you use this software in a product, an acknowledgment
%    in the product documentation would be appreciated but is not required.
%
% 2. Altered source versions must be plainly marked as such,
%    and must not be misrepresented as being the original software.
%
% 3. This notice may not be removed or altered from any source distribution.
%
%%%%%%%%%%%%%%%%%%%%%%%%%%%%%%%%%%%%%%%%%%%%%%%%%%%%%%%%%%%%%%%%%%%%%%%%%%%%%%%%%%%%%%%%%%%%%%%%%%%

\section{\texttt{minus\_assignment\_functor} (struct)}
A \textit{FunctorBase} (see \cite{functors}) with two template parameters \texttt{A} and \texttt{B} followed by the \texttt{ENABLED} parameter.
Its specializations are \textit{Functor}s representing the binary operation $\texttt{a -= b}$.

\subsection{\texttt{minus\_assignment} (function)}
This library provides a function to invoke the \texttt{minus\_assignment\_functor}.
A possible implementation is given by\newline
\texttt{template\textlangle typename A, typename B\textrangle\newline
auto\newline
minus\_assignment(A\& a, const B\& b)\newline
-> decltype(minus\_assignment\_functor\textlangle A, B\textrangle()(a, b))\newline
\{ return minus\_assignment\_functor\textlangle A, B\textrangle()(a, b); \}}

\subsection{\texttt{operator+=} (compound operator)}
This library provides a compound minus operator overload to invoke the \texttt{minus\_assignment\_functor}.
A possible implementation is given by\newline
\texttt{template\textlangle typename A, typename B\textrangle\newline
auto\newline
operator-=(A\& a, const B\& b)\newline
-> decltype(minus\_assignment(a, b))\newline
\{ return minus\_assignment(a, b); \}}

%%%%%%%%%%%%%%%%%%%%%%%%%%%%%%%%%%%%%%%%%%%%%%%%%%%%%%%%%%%%%%%%%%%%%%%%%%%%%%%%%%%%%%%%%%%%%%%%%%%
\section{\texttt{star\_assignment\_functor} (struct)}
A \textit{AssignmentFunctorBase} representing the binary operation $\textit{left\_operand} += \textit{right\_operand}$.\newline

\noindent{}This library provides \textit{AssignmentFunctorBase} specializations for
\texttt{LEFT\_OPERAND} and \texttt{RIGHT\_OPERAND} being the same    floating point
type which compute  the sum of  \texttt{left\_operand} and \texttt{right\_operand}
and assigns the result to \texttt{left\_operand}.
\texttt{result\_type} is \texttt{left\_operand}.

\subsection{\texttt{star\_assignment} (function)}
This library provides a function to invoke the \texttt{star\_assignment\_functor}.
A possible implementation is given by\newline
\texttt{
template\textlangle typename A, typename B\textrangle\newline
auto star\_assignment(A\& a, const B\& b) -> decltype(star\_assignment\_functor\textlangle A, B\textrangle()(a, b))\newline
\{ return star\_assignment\_functor\textlangle A, B\textrangle()(a, b); \}
}

\subsection{\texttt{operator*=} (compound operator)}
This library provides a compound star operator overload to invoke the \texttt{star\_assignment\_functor}.
A possible implementation is given by\newline
\texttt{
template\textlangle typename A, typename B\textrangle\newline
auto operator*=(A\& a, const B\& b) -> decltype(star\_assignment(a, b))\newline
\{ return star\_assignment(a, b); \}
}
%%%%%%%%%%%%%%%%%%%%%%%%%%%%%%%%%%%%%%%%%%%%%%%%%%%%%%%%%%%%%%%%%%%%%%%%%%%%%%%%%%%%%%%%%%%%%%%%%%%

\section{\texttt{slash\_assignment\_functor} (struct)}
A \textit{FunctorBase} (see \cite{functors}) with two template parameters \texttt{A} and \texttt{B} followed by the \texttt{ENABLED} template parameter.
Its specializations are \textit{Functor}s representing the operation $\textit{a /= b}$.

\subsection{\texttt{slash\_assignment} (function)}
This library provides a function to invoke the \texttt{slash\_assignment\_functor}.
A possible implementation is given by\newline
\texttt{template\textlangle typename A, typename B\textrangle\newline
auto\newline
slash\_assignment(A\& a, const B\& b)\newline
-> decltype(slash\_assignment\_functor\textlangle A, B\textrangle()(a, b))\newline
\{ return slash\_assignment\_functor\textlangle A, B\textrangle()(a, b); \}}

\subsection{\texttt{operator/=} (compound operator)}
This library provides a compound slash operator overload to invoke the \texttt{slash\_assignment\_functor}.
A possible implementation is given by\newline
\texttt{template\textlangle typename A, typename B\textrangle\newline
auto\newline
operator/=(A\& a, const B\& b)\newline
-> decltype(slash\_assignment(a, b))\newline
\{ return slash\_assignment(a, b); \}}

%%%%%%%%%%%%%%%%%%%%%%%%%%%%%%%%%%%%%%%%%%%%%%%%%%%%%%%%%%%%%%%%%%%%%%%%%%%%%%%%%%%%%%%%%%%%%%%%%%%
\section{\texttt{division\_by\_zero\_error} (struct)}
An error (as specified in Primordial Machine's Errors library) indicating a division by zero.

\section{\textit{Error} (concept)}
\textit{\color{orange}This section needs to be moved in Primordial Machine's Error library.}
\noindent\textit{\color{orange}[[}\\
An \textit{Error} type is a \texttt{error} or derived type.\\
\noindent\textit{\color{orange}]]}

\section{\textit{Exception} (concept)}
\textit{\color{orange}This section needs to be moved in Primordial Machine's Error library.}
\noindent\textit{\color{orange}[[}\\
An \textit{Exception} type is a \texttt{exception} or derived type.
Such an object must retain the properties of being
\begin{enumerate}
  \item noexcept copy/move constructible and
  \item noexcept copy/move assignable.
\end{enumerate}
\noindent\textit{\color{orange}]]}

\section{\texttt{negative\_overflow\_error} (struct)}
An \textit{Error} type (as specified in Primordial Machine's Errors library) indicating a negative overflow.
It raises its corresponding \texttt{negative\_overflow\_exception}.
The term negative overflow refers to the situation in which an arithmetic value is non-positive and
too big (in terms of its magnitude) to be represented. Its constructor takes a single argument which
is a \texttt{error\_position} value.

\section{\texttt{positive\_overflow\_error} (struct)}
An \textit{Error} type (as specified in Primordial Machine's Errors library) indicating a positive overflow.
It raises its corresponding \texttt{negative\_overflow\_exception}.
The term positive overflow refers to the situation in which an arithmetic value is non-negative and
too big (in terms of its magnitude) to be represented. Its constructor takes a single argument which
is a \texttt{error\_position} value.

\section{\texttt{underflow\_error} (struct)}
An \textit{Error} type (as specified in Primordial Machine's Errors library) indicating an underflow.
It raises its corresponding \texttt{underflow\_exception}.
The term underflow refers to the situation in which an arithmetic value is too small (in terms of
its magnitude) to be represented. Its constructor takes a single argument which
is a \texttt{error\_position} value.

\end{document}
