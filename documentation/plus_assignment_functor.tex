%%%%%%%%%%%%%%%%%%%%%%%%%%%%%%%%%%%%%%%%%%%%%%%%%%%%%%%%%%%%%%%%%%%%%%%%%%%%%%%%%%%%%%%%%%%%%%%%%%%
\section{\texttt{plus\_assignment\_functor} (struct)}
A \textit{AssignmentFunctorBase} representing the binary operation $\textit{left\_operand} += \textit{right\_operand}$.\newline

\noindent{}This library provides \textit{AssignmentFunctorBase} specializations for
\texttt{LEFT\_OPERAND} and \texttt{RIGHT\_OPERAND} being the same    floating point
type which compute  the sum of  \texttt{left\_operand} and \texttt{right\_operand}
and assigns the result to \texttt{left\_operand}.
\texttt{result\_type} is \texttt{left\_operand}.

\subsection{\texttt{plus\_assignment} (function)}
This library provides a function to invoke the \texttt{plus\_assignment\_functor}.
A possible implementation is given by\newline
\texttt{
template\textlangle typename A, typename B\textrangle\newline
auto plus\_assignment(A\& a, const B\& b) -> decltype(plus\_assignment\_functor\textlangle A, B\textrangle()(a, b))\newline
\{ return plus\_assignment\_functor\textlangle A, B\textrangle()(a, b); \}
}

\subsection{\texttt{operator+=} (compound operator)}
This library provides a compound plus operator overload to invoke the \texttt{plus\_assignment\_functor}.
A possible implementation is given by\newline
\texttt{
template\textlangle typename A, typename B\textrangle\newline
auto operator+=(A\& a, const B\& b) -> decltype(plus\_assignment(a, b))\newline
\{ return plus\_assignment(a, b); \}
}
